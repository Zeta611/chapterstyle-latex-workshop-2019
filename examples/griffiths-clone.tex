\documentclass[openany]{memoir}

% Page layout settings
\setstocksize{237mm}{179mm}
\settrimmedsize{\stockheight}{\stockwidth}{*}
\setlrmarginsandblock{146pt}{39pt}{*}
\setulmarginsandblock{60pt}{76pt}{*}
\setheaderspaces{*}{13pt}{*}
\checkandfixthelayout
\setlength{\evensidemargin}{\oddsidemargin}

% Font settings
\usepackage{amsmath}
\usepackage[lite,subscriptcorrection,slantedGreek,nofontinfo,amsbb,eucal]{mtpro2}

\usepackage[no-math]{fontspec}
\setmainfont{Times New Roman}
\newfontfamily\optimafont{Optima}

% Custom commands
\usepackage{float}
\usepackage{lipsum}
\usepackage{amssymb}
\usepackage{xparse}
\usepackage{graphicx}

\def\rcurs{\mbox{$\resizebox{.115in}{.067in}{\includegraphics{ScriptR}}$}}
\def\brcurs{\mbox{$\resizebox{.15in}{.07in}{\includegraphics{BoldR}}$}}
\def\hrcurs{\mbox{$\hat \brcurs$}}

\renewcommand*{\vec}[1]{\mathbf{#1}}
\newcommand*{\shiftleft}[2]{\makebox[0pt][r]{\makebox[#1][l]{#2}}}
\newcommand*{\shiftright}[2]{\makebox[#1][r]{\makebox[0pt][l]{#2}}}

% Pagestyle settings
\makepagestyle{griffiths}
\makeatletter
\makepsmarks{griffiths}{%
  \nouppercaseheads
  \createmark{chapter}{left}{shownumber}{\@chapapp\ }{\quad}
  \createmark{section}{right}{shownumber}{}{\quad}}
\makeatother

\makeevenhead{griffiths}{%
  \optimafont\shiftleft{107pt}{\bfseries\thepage}\small\leftmark}{}{}
\makeoddhead{griffiths}{\optimafont\small\rightmark}
                       {}{\optimafont\bfseries\thepage}

\makepagestyle{gchapter}
\makeevenfoot{gchapter}{\optimafont\shiftleft{107pt}{\bfseries\thepage}}{}{}
\makeoddfoot{gchapter}{}{}{\optimafont\bfseries\thepage}

% Chapterstyle settings
\makeatletter
\newlength{\chapappwidth}
\makechapterstyle{griffiths}{%
  \chapterstyle{default}
  \setlength\beforechapskip{-42pt}
  \renewcommand*{\afterchapternum}{\qquad}
  \setlength\afterchapskip{63pt}
  \renewcommand*{\chapnamefont}{%
    \optimafont\bfseries\addfontfeatures{LetterSpace=26}\huge}
  \def\chaptertext{\chapnamefont\MakeUppercase{\@chapapp}}
  \settowidth{\chapappwidth}{\chaptertext}
  \renewcommand*{\printchaptername}{%
    \begin{adjustwidth}{-\chapappwidth}{}
      \chaptertext
    \end{adjustwidth}}
  \renewcommand*{\chapternamenum}{\vskip 12pt}
  \renewcommand*{\chapnumfont}{%
    \optimafont\bfseries\fontsize{60}{60}\selectfont}
  \renewcommand*{\printchapternum}{}
  \renewcommand*{\chaptitlefont}{%
    \optimafont\bfseries\fontsize{22}{22}\selectfont}
  \renewcommand*{\printchaptertitle}[1]{%
    \begin{adjustwidth}{-\chapappwidth}{}
      \makebox[\chapappwidth][c]{\chapnumfont\thechapter}%
      \makebox[\dimexpr\linewidth-\chapappwidth\relax][c]{\chaptitlefont ##1}
    \end{adjustwidth}}}
\makeatother
\renewcommand*{\memendofchapterhook}{\thispagestyle{gchapter}}

% Sectionstyle settings
\setsecnumdepth{subsection}
\maxsecnumdepth{subsection}

\setsecnumformat{\llap{\csname the#1\endcsname\ $\blacksquare$\ }}
\setsecheadstyle{\optimafont\bfseries\MakeUppercase}
\setsubsecheadstyle{\optimafont\bfseries}

% Footnote settings
\setlength{\footmarkwidth}{0pt}
\setlength{\footmarksep}{0pt}
\let\oldfootnoterule\footnoterule
\renewcommand*{\footnoterule}{}


\begin{document}
\setcounter{page}{59}
\setcounter{chapter}{1}
\pagestyle{griffiths}
\chapterstyle{griffiths}

\chapter{Electrostatics}
\section{The Electric Field}
\subsection{Introduction}
The fundamental problem electrodynamics hopes to solve is this (Fig. 2.1):
We have some electric charges, $q_1, q_2, q_3, \dots$ (call them
\strong{source charges});
what force do they exert on another charge, $Q$ (call it the
\strong{test charge})?
The positions of the source charges are \emph{given} (as functions of time);
the trajectory of the test particle is \emph{to be calculated}.
In general, both the source charges and the test charge are in motion.

The solution to this problem is facilitated by the
\strong{principle of superposition}, which states that the interaction between
any two charges is completely unaffected by the presence of others.
This means that to determine the force on $Q$, we can first compute the force
$\vec{F}_1$, due to $q_1$ alone (ignoring all the others);
then we compute the force $\vec{F}_2$, due to $q_2$ alone;
and so on.
Finally, we take the vector sum of all these individual forces:
$\vec{F} = \vec{F}_1 + \vec{F}_2 + \vec{F}_3 + \dots$
Thus, if we can find the force on $Q$ due to a \emph{single} source charge $q$,
we are, in principle, done (the rest is just aquestion of repeating the same
operation over and over, and adding it all up).\footnote{The principle of
  superposition may seem ``obvious'' to you, but it did not have to be so
  simple:
  if the electromagnetic force were proportional to the square of the total
  source charge, for instance, the principle of superposition would not hold,
  since $(q_1 + q_2)^2 \neq q_1^2 + q_2^2$ (there would be ``cross terms'' to
  consider).
  Superposition is not a logical necessity, but an experimental fact.}

Well, at first sight this looks very easy:
Why don’t I just write down the formula for the force on $Q$ due to $q$, and
be done with it?
I \emph{could}, and in Chapter 10 I shall, but you would be shocked to see it
at this stage, for not only does the force
\makebox[\linewidth][s]{%
on $Q$ depend on the separation distance \rcurs between the charges (Fig. 2.2),
it also}
\begin{figure}[H]
  \centering
  \includegraphics[width=\linewidth]{figure}
\end{figure}

\lipsum[1]

\subsection{Coloumb's Law}
\lipsum[2-5]

\subsection{The Electric Field}
\lipsum[6-9]
\end{document}
